\documentclass[a4paper,11pt]{article}
\usepackage[utf8]{inputenc}
\usepackage[frenchb]{babel}
\usepackage{amssymb}
\usepackage{amsmath}
\usepackage{amsthm}
\usepackage{mathrsfs}
\usepackage{array}
\usepackage{graphicx}
\usepackage[usenames,dvipsnames]{color}
\usepackage{listings}
%\usepackage{arydshln}
%\usepackage{slashbox}
\usepackage{subfigure}
%\usepackage{cancel}
%\usepackage[bookmarks = false]{hyperref}
\usepackage[left=2cm, right=2cm, top=2cm, bottom=2cm]{geometry}

% Initialisation de listings
%\definecolor{mymauve}{rgb}{0.63,0.13,0.94}
%\definecolor{mygreen}{rgb}{0.13,0.55,0.13}
%\definecolor{mybeige}{rgb}{0.99,0.99,0.86}
%\definecolor{mygris}{rgb}{0.8,0.8,0.8}
\definecolor{light-gray}{gray}{0.50}
\lstset{
    columns=flexible,
	%numbers = left,				% placement de la numérotation des lignes
	numberstyle = \small,        	% taille du numéro de ligne
	stepnumber = 1,              	% ???
	numbersep = 10pt,            	% taille de l'espace de séparation entre numéro de ligne et code
	showspaces = false,          	% montrer les espaces
	showstringspaces=false,         % enlever les espaces str
	showtabs = false,            	% montrer les tabulations
	tab = rightarrowfill,        	% ???
	tabsize=3,						% tabulation size
	language = Java,             	% langage utilisé
	basicstyle = \footnotesize\tt,	% ???
	captionpos = b,					% ???
	linewidth=\linewidth,			% largeur de la fenetre de code
	breaklines = true,				% ???
	commentstyle = \color{light-gray}, % définition de la couleur des commentaires
	%stringstyle = \color{mymauve},  % définition de la couleur des chaines de caractères
	%identifierstyle = \ttfamily,    % ???
	keywordstyle = \color{blue},	% définition de la couleur des mots clés
	%frame=single,
	%backgroundcolor=\color{mybeige},
	extendedchars=true				% étend les caractères pouvant être utilisés
}

%\author{Mormont Romain}
%\title{Synthèse : Base de données (Pierre Wolper)}
%\date{Année académique 2013-2014}

\begin{document}
\rule{1\linewidth}{1px}
{ \sc
\begin{center}
{\small Université de Liège}\\
{\small Faculté des Sciences Appliquées}

\end{center}

\vfill
\begin{center}

{\Huge Introduction to computer networking {\LARGE \tt [INFO0010-1]}\\}
\end{center}
\begin{center}
{\Huge Projet 2 : Rapport}
\end{center}
\begin{center}
Mormont Romain, Servais Fabrice\\
{\small 3$^{\text{ème}}$  bachelier ingénieur civil, orientation ingénieur civil}\\
{\small Options \textit{informatique} et \textit{électricité et électronique}}\\
{\small s110940, s111093}
\end{center}

\vfill
\begin{center}
Année académique 2013-2014\\
\end{center}
}
\rule{1\linewidth}{1px}
\newpage

\section{Software architecture}
Le problème a été divisé en différents sous-problèmes que nous allons expliciter dans leur ordre temporel.

	\subsection{Initialisation du Gateway}
Une instance de \texttt{Server} est créée au lancement de cette classe. Celle-ci crée une instance de \texttt{HTTPServer} et \texttt{ConfigurationServer} afin d'accepter les connexions entrantes vers le Gateway et la plate-forme de configuration. Elle récupère aussi le singleton \texttt{Displayer} servant à afficher des messages dans la console. Nous ne détaillerons pas la plate-forme de configuration.

	\subsection{Récupération des connexions entrantes}
La classe \texttt{HTTPServer} récupère les connexions entrantes et lance un thread (si possible par la thread-pool) pour chacune d'entre-elles par l'intermédiaire de la classe \texttt{HTTPClientRequestThread} et ce, après l'acceptation de connexion du socket.

	\subsection{Gestion des requêtes}





\end{document}